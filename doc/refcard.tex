% mpatrol
% A library for controlling and tracing dynamic memory allocations.
% Copyright (C) 1997-2000 Graeme S. Roy <graeme@epc.co.uk>
%
% This library is free software; you can redistribute it and/or
% modify it under the terms of the GNU Library General Public
% License as published by the Free Software Foundation; either
% version 2 of the License, or (at your option) any later version.
%
% This library is distributed in the hope that it will be useful,
% but WITHOUT ANY WARRANTY; without even the implied warranty of
% MERCHANTABILITY or FITNESS FOR A PARTICULAR PURPOSE.  See the GNU
% Library General Public License for more details.
%
% You should have received a copy of the GNU Library General Public
% License along with this library; if not, write to the Free
% Software Foundation, Inc., 59 Temple Place, Suite 330, Boston,
% MA 02111-1307, USA.

% LaTeX reference card for mpatrol

% $Id: refcard.tex,v 1.1 2000-05-12 18:37:26 graeme Exp $

\documentclass[a4paper,landscape,final]{article}

\usepackage{multicol}

% Adapt the page dimensions to the paper size.

\setlength{\textwidth}{\paperwidth}
\addtolength{\textwidth}{-1in}
\setlength{\textheight}{\paperheight}
\addtolength{\textheight}{-1in}
\setlength{\oddsidemargin}{-.5in}
\setlength{\evensidemargin}{-.5in}
\setlength{\topmargin}{-.5in}
\setlength{\headheight}{0in}
\setlength{\headsep}{0in}
\setlength{\footskip}{0in}
\setlength{\parindent}{0in}
\setlength{\columnsep}{.5in}

% Define new commands for formatting the headings and options.

\newcommand{\heading}[1]{\textbf{\normalsize #1}}
\newcommand{\option}[1]{\texttt{#1}}
\newcommand{\optionarg}[2]{\option{#1}\texttt{=<#2>}}
\newcommand{\optionpar}[2]{\option{#1}\texttt{=<}\textit{#2}\texttt{>}}

\begin{document}

\pagestyle{empty}

\footnotesize

\begin{multicols}{3}{\textbf{\Large mpatrol reference card}}

\vskip 12pt

The mpatrol library can read certain options at run-time from an environment
variable called \texttt{MPATROL\_OPTIONS}.  This variable must contain one or
more valid option keywords from the list below and must be no longer than 1024
characters in length.  If \texttt{MPATROL\_OPTIONS} is unset or empty then the
default settings will be used.

\vskip 12pt
\heading{Library behaviour}
\vskip 6pt

\begin{description}
\item[\option{HELP}]
Displays a quick-reference option summary.
\item[\optionpar{PROGFILE}{string}]
Specifies an alternative filename with which to locate the executable file
containing the program's symbols.
\item[\optionpar{CHECK}{unsigned range}]
Specifies a range of allocation indices at which to check the integrity of free
memory and overflow buffers.
\item[\optionpar{DEFALIGN}{unsigned integer}]
Specifies the default alignment for general-purpose memory allocations, which
must be a power of two.
\item[\option{NOPROTECT}]
Specifies that the mpatrol library's internal data structures should not be made
read-only after every memory allocation, reallocation or deallocation.
\item[\option{SAFESIGNALS}]
Instructs the library to save and replace certain signal handlers during the
execution of library code and to restore them afterwards.
\item[\option{USEMMAP}]
Specifies that the library should use \texttt{mmap()} instead of \texttt{sbrk()}
to allocate system memory on UNIX platforms.
\end{description}

\vskip 12pt
\heading{Logging and tracing}
\vskip 6pt

\begin{description}
\item[\optionpar{LOGFILE}{string}]
Specifies an alternative file in which to place all diagnostics from the mpatrol
library.
\item[\option{LOGALLOCS}]
Specifies that all memory allocations are to be logged and sent to the log file.
\item[\option{LOGREALLOCS}]
Specifies that all memory reallocations are to be logged and sent to the log
file.
\item[\option{LOGFREES}]
Specifies that all memory deallocations are to be logged and sent to the log
file.
\item[\option{LOGMEMORY}]
Specifies that all memory operations are to be logged and sent to the log file.
\item[\option{LOGALL}]
Equivalent to the \option{LOGALLOCS}, \option{LOGREALLOCS}, \option{LOGFREES}
and \option{LOGMEMORY} options specified together.
\item[\option{SHOWFREED}]
Specifies that a summary of all of the freed memory allocations should be
displayed at the end of program execution.
\item[\option{SHOWUNFREED}]
Specifies that a summary of all of the unfreed memory allocations should be
displayed at the end of program execution.
\item[\option{SHOWMAP}]
Specifies that a memory map of the entire heap should be displayed at the end of
program execution.
\item[\option{SHOWSYMBOLS}]
Specifies that a summary of all of the function symbols read from the program's
executable file should be displayed at the end of program execution.
\item[\option{SHOWALL}]
Equivalent to the \option{SHOWFREED}, \option{SHOWUNFREED}, \option{SHOWMAP} and
\option{SHOWSYMBOLS} options specified together.
\item[\option{USEDEBUG}]
Specifies that any debugging information in the executable file should be used
to obtain additional source-level information.
\end{description}

\vskip 12pt
\heading{General errors}
\vskip 6pt

\begin{description}
\item[\option{CHECKALLOCS}]
Checks that no attempt is made to allocate a block of memory of size zero.
\item[\option{CHECKREALLOCS}]
Checks that no attempt is made to reallocate a \texttt{NULL} pointer or resize
an existing block of memory to size zero.
\item[\option{CHECKFREES}]
Checks that no attempt is made to deallocate a \texttt{NULL} pointer.
\item[\option{CHECKALL}]
Equivalent to the \option{CHECKALLOCS}, \option{CHECKREALLOCS} and
\option{CHECKFREES} options specified together.
\item[\optionpar{ALLOCBYTE}{unsigned integer}]
Specifies an 8-bit byte pattern with which to prefill newly-allocated memory.
\item[\optionpar{FREEBYTE}{unsigned integer}]
Specifies an 8-bit byte pattern with which to prefill newly-freed memory.
\item[\option{NOFREE}]
Specifies that the mpatrol library should keep all reallocated and freed memory
allocations.
\item[\option{PRESERVE}]
Specifies that any reallocated or freed memory allocations should preserve their
original contents.
\end{description}

\vskip 12pt
\heading{Overwrites and underwrites}
\vskip 6pt

\begin{description}
\item[\optionpar{OFLOWSIZE}{unsigned integer}]
Specifies the size in bytes to use for all overflow buffers, which must be a
power of two.
\item[\optionpar{OFLOWBYTE}{unsigned integer}]
Specifies an 8-bit byte pattern with which to fill the overflow buffers of all
memory allocations.
\item[\option{OFLOWWATCH}]
Specifies that watch point areas should be used for overflow buffers rather than
filling with the overflow byte.
\item[\optionarg{PAGEALLOC}{LOWER|UPPER}]
Specifies that each individual memory allocation should occupy at least one page
of virtual memory and should be placed at the lowest or highest point within
these pages.
\end{description}

\vskip 12pt
\heading{Using with a debugger}
\vskip 6pt

\begin{description}
\item[\optionpar{ALLOCSTOP}{unsigned integer}]
Specifies an allocation index at which to stop the program when it is being
allocated.
\item[\optionpar{REALLOCSTOP}{unsigned integer}]
Specifies an allocation index at which to stop the program when a memory
allocation is being reallocated.
\item[\optionpar{FREESTOP}{unsigned integer}]
Specifies an allocation index at which to stop the program when it is being
freed.
\end{description}

\vskip 12pt
\heading{Testing}
\vskip 6pt

\begin{description}
\item[\optionpar{LIMIT}{unsigned integer}]
Specifies the limit in bytes at which all memory allocations should fail if the
total allocated memory should increase beyond this.
\item[\optionpar{FAILFREQ}{unsigned integer}]
Specifies the frequency at which all memory allocations will randomly fail.
\item[\optionpar{FAILSEED}{unsigned integer}]
Specifies the random number seed which will be used when determining which
memory allocations will randomly fail.
\item[\optionpar{UNFREEDABORT}{unsigned integer}]
Specifies the minimum number of unfreed allocations at which to abort the
program just before program termination.
\end{description}

\vskip 12pt
\heading{Profiling}
\vskip 6pt

\begin{description}
\item[\option{PROF}]
Specifies that all memory allocations are to be profiled and sent to the
profiling output file.
\item[\optionpar{PROFFILE}{string}]
Specifies an alternative file in which to place all memory allocation profiling
information from the mpatrol library.
\item[\optionpar{AUTOSAVE}{unsigned integer}]
Specifies the frequency at which to periodically write the profiling data to the
profiling output file.
\item[\optionpar{SMALLBOUND}{unsigned integer}]
Specifies the limit in bytes up to which memory allocations should be classified
as small allocations for profiling purposes.
\item[\optionpar{MEDIUMBOUND}{unsigned integer}]
Specifies the limit in bytes up to which memory allocations should be classified
as medium allocations for profiling purposes.
\item[\optionpar{LARGEBOUND}{unsigned integer}]
Specifies the limit in bytes up to which memory allocations should be classified
as large allocations for profiling purposes.
\end{description}

\vskip 12pt
Copyright \copyright 1997-2000 Graeme S. Roy.
\vskip 6pt

This reference card may be freely distributed under the terms of the GNU General
Public License.

\end{multicols}

\end{document}
